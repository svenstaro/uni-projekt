%fazit

\section{Ergebnisse}
% voll funktionsfähiger prototyp
% sogar mit fpga ding
% 'über das ziel hinausgeschossen'

\section{Probleme} %0ortmann

\subsection{Gruppe \todo{hier evtl nicht gliedern}}
Zu Beginn unseres Projektes bestand die Gruppe aus 7 Mitgliedern, die sich nach recht kurzer Zeit um zwei reduziert hat. Schade war, dass die Mitglieder nicht einfach ausgetreten sind, sondern bis zum Ende trotz wiederholten Anfragen einfach nichts getan haben. Im Folgenden ist mit \textit{unserer Gruppe} also das reduzierte Team von 5 Leuten gemeint.
Im Verlauf underes Projektes haben sich über die Zeit einige Probleme ergeben. Dies war meist begründet durch kurzfristig nötige Programmänderungen, die dann zu größeren Diskussionen führten. Im Allgemenein hat unsere Gruppe dazu geneigt, Einzelheiten sehr fein auszudiskutieren. Auch bei der vielfältigen Änderung unserer ISA, siehe auch das zugehörige Kapitel, hat sich einiges an Streitigkeiten ergeben. Besonders anzumerken ist jedoch, dass alle Streitigkeiten in unserer Gruppe stets fachlich begründet waren. Unsere Mitglieder haben sich nicht etwa aus persönlichen Gründen nicht verstanden, sondern hatten vielmehr alle immer konstruktiv etwas zu den Fragestellungen beizutragen und daraus ergaben sich dann größere Diskussionen.

\subsection{Fachliche Probleme}
Wie auch schon im Kapitel ISA kurz angerissen wurde, ergaben sich vielfältige Änderungen an der Befehlsstruktur. Zunächst hatten wir nur ein sehr begrenztes Befehlset, am Ende auch Befehle zum direkten Ansteuren von Hardwareelementen. Die Problematik dahinter lag in der Hardware des FPGA begründet. 
Ein bis jetzt bestehendes Problem unserer Architektur ist, dass unser Microcontroller kein byteweises Laden und Speichern unterstüztz. Auch Kommunikationsmanagement bzw. Nachrichtenaustauschürotokolle unterstrützt der Controller nicht und dementsprechend kennt er auch keine Interrupts.


% isa musste mehrfach angepasst werden
% |- weil moppelkotze
% |- weil hardware dumm
% bestehende probleme
% |- kein byteweises laden/speichern
% intergruppendynamikkommunikationsaustauschverhalten
% sven und alisa sind rausgeflogen

\section{Erkenntnisse}
% TODO: hardware designen ist nicht schwer
% man muss sich nur gedanken machen
% verständniss für hardware wurde verbessert
% verständniss für software wurde verbessert
% gruppenarbeit ist wichtig und toll
% issue tracker nutzen um übersicht zu behalten
% git ist toll

\section{Ausblick} %1hellwig
% was könnten wir noch machen
% |- video
% |- audio
% |- pipeline
% |- cache
% |- interrupts
% |- multi threading ...
% assembler (siehe github)

\section{Fazit}
% TODO: war super
% machen wir gern noch mal
% wir sehen uns im master
