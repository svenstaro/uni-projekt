\documentclass[a4paper]{article}

\usepackage[ngerman]{babel}
\usepackage[utf8]{inputenc} 
\usepackage{amsmath}
\usepackage{amssymb}
\usepackage{fancyhdr}
\usepackage{graphicx}
\usepackage{pdfpages}
\usepackage[left=2cm,right=2cm,top=2cm,bottom=3cm,includeheadfoot]{geometry}
\usepackage[scaled=.90]{helvet}
\usepackage{courier}
\usepackage{dsfont} 
\usepackage{multicol}
\usepackage{tikz}
\usetikzlibrary{arrows,shapes,automata,backgrounds,petri}
\usepackage{listings} \lstset{numbers=left, numberstyle=\tiny, numbersep=5pt} \lstset{language=Perl} 

\pagestyle{fancy} %eigener Seitenstil
\fancyhf{} %alle Kopf- und Fußzeilenfelder bereinigen
%\fancyhead[L]{\doctitle  \nummer} %Kopfzeile links
\fancyhead[L]{\docauthors} %Kopfzeile rechts
\renewcommand{\headrulewidth}{0.5pt} %obere Trennlinie
\fancyfoot[L]{\doctitle  \nummer} %Kopfzeile links
\fancyfoot[R]{Seite \thepage} %Fußzeile rechts
\renewcommand{\footrulewidth}{0.5pt} %untere Trennlinie

% Titelseite Block oben
\newcommand{\titelblock}{
\sloppy
\begin{center}
\sffamily
{\Large{\veranstaltung \\}}
{\Huge{\doctitle  \nummer}\\}
\vspace{0.5cm}
\tutorium \\
\hrulefill
\end{center}
}

\newcommand{\docauthors}{Felix Wiedemann, Marcel Hellwig, Nasif Yueksel, David Weber, Sven-Hendrik Haase, Alias Dammer, Felix Ortmann}
\newcommand{\docdate}{\today}
\newcommand{\doctitle}{Ideensammlung}
\newcommand{\nummer}{ 1} %Übungszettelnummer hier anpassen
\newcommand{\tutorium}{WS 2013/2014}
\newcommand{\veranstaltung}{Projekt: Mikrorechner}
\newcommand{\gruppe}{}


\begin{document}
\titelblock

\section{Ideensammlung - Instruction Set Architecture}

\subsection{General}

\begin{multicols}{2}
\begin{itemize}
  \item Wortbreite 32 bit
  \item Adressbreite 32 bit
  \item 3-Operanden (1 Ziel-, 2 Quell-Operanden)
  \item 13 General-Purpose Register
  \item \$0-Register: Null
  \item \$14-Register: Stack Pointer
  \item \$15-Register: Return Address
  \item 4 Statusflags
  \begin{itemize}
    \item[Z] Zero - 1 wenn Ergebnis == 0
    \item[N] Negative - 1 wenn MSB == 1
    \item[C] Carry - 1 wenn ein (signed) Over- oder Underflow stattgefunden hat
    \item[V] Overflow - 1 wenn ein (unsigned) Over- oder Underflow stattgefunden hat
  \end{itemize}
  \item Big Endian
  \item Befehle können Statusflags setzen
  \item 16-Bit Immediates
\end{itemize}

\subsection{Arithmetik / Logik}

\begin{itemize}
\item ADD, ADC
\item SUB, SBC, RSB, RSC
\item MUL, DIV
\item LSL, ADR, LSR, ROR 
\item AND, ORR, XOR, NOT
\item SWP (Wenn Felix ganz lieb fragt evtl soch)
\end{itemize}

\subsection{Controlflow}

\begin{itemize}
\item JMP
\item Jxx (Flag dependent JMP)
\item CALL
\item RET (== JMP \$15)
\item SWI (Software Interrupt)
\end{itemize}

\subsection{Memory}

\begin{itemize}
\item LD
\item STR
\end{itemize}
\end{multicols}

%\newline
%\hline

\subsection{Befehlsstruktur}
\[ALU - 00 : Opcode - (4bit) : SF - (1bit) : R_{dest} - (4bit) : R_{s1} - (4bit) : OP2 - (17bit)\]
\[MEM - 10 : LD/ST - (1bit) : R_{dest} - (4bit) : OP2 - (25bit)\]
\[Jxx - 110 : Condition - (4bit) : OP2 - (25bit)\]
\[JMP - 1110000 : OP2 - (25bit)\]
\[CALL- 1110001 : OP2 - (25bit)\]
\[ADR - 010 : R_{dest} - (4bit) : OP2 - (25bit)\]
\[SWI - 0110010 : OP2 - (25bit)\]
\[PUSH- 0110100 : 0* : R_{src} - (4bit)\]
\[POP - 0110101 : 0* : R_{dest} - (4bit)\]

\subsection{OP2}
\[ 0 : R_{src} - (4bit) : 0* \]
\[ 1 : Immediate - (size-1 bit) \]

\subsection{Fancy Magic}

\begin{itemize}
\item Divisionseinheit
\item Fließkomma
\item Touchscreen / Display (Mäder: das is ganz eklig) - flieht ihr Narren!!
\item Cache
\item SIMD
\end{itemize}


\subsection{Other Stuff, nice-to-haves, not-to-be-dones}

\begin{itemize}
\item read / write from somewhere
\item Software Interrupts
\item MMU (Kommentar aus dem Plenum: unnötig und scheiße)
\item OS-Aufrufe (falls wir mal eins haben)
\item GCC und lib-C portieren
\item Linux portieren
\item SD-Karte nutzen (wie Rasp-Pi?? als cache??)
\end{itemize}

\subsection{Codierung der Befehle - Binär}
Arithmetisch/Logische Befehle fangen mit 00 an
\begin{multicols}{2}
\emph{Präfix 0 + {Opcode}}
\begin{itemize}
\item ADD 000
\item ADC 001
\item SUB 100
\item SBC 101
\item RSB 110
\item RSC 111

\item MUL 010
\item DIV 011
\end{itemize}

\emph{Präfix 10 + {Opcode}}
\begin{itemize}
\item AND 00
\item ORR 01
\item XOR 10
\item NOT 11
\end{itemize}

\emph{Präfix 11 + {Opcode}}
\begin{itemize}
\item LSL 00
\item ASR 01
\item LSR 10
\item ROR 11
\end{itemize}


\end{multicols}



\end{document}
