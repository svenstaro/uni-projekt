\textit{GPLv3, GNU General Public License version 3} \newline
Von Beginn an haben wir unser Projekt weltöffentlich und zugänglich auf github gemacht. Im Zuge dieser Art der Veröffentlichung erschien es uns nötig, den Code unter Lizenz zu stellen. Es gibt zu genau diesem Zweck viele verschiedene Arten von Lizenzen; viele haben den Zweck den tatsächlichen Autor des Codes davor zu schützen, dass Dritte diese Arbeit verkaufen und den Autor ausbeuten. Dabei darf jedoch nicht außer acht gelassen werden, dass der Code open source ist und jedem zugänglich sein \textit{soll}. \newline
Wir haben uns in unserem Projekt für die Lizenz \textit{GNU General Public License version 3} entschieden. Damit sind vorallem die Nutzer unserer Software nicht benachteiligt, die Nutzung ist uneingeschränkt möglich und erlaubt. Der Download, das Verteilen und auch die Veränderung zu eigenen Zwecken werden in der Lizenz behandelt und explizit erlaubt. Was jedoch nicht erlaubt ist, ist die kommerzielle Distribution unseres Codes ohne uns als Autoren zu informieren und zu beteiligen. Es lässt sich festhalten, dass die Lizenzsierung hauptsächlich aus dem Wunsch heraus motiviert wurde, unseren Code nicht an einen unbekannten Dritten rechtlich zu verlieren.
