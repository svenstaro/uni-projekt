\documentclass[a4paper]{article}

\usepackage[ngerman]{babel}
\usepackage[utf8]{inputenc} 
\usepackage{amsmath}
\usepackage{amssymb}
\usepackage{fancyhdr}
\usepackage{graphicx}
\usepackage{pdfpages}
\usepackage[left=2cm,right=2cm,top=2cm,bottom=3cm,includeheadfoot]{geometry}
\usepackage[scaled=.90]{helvet}
\usepackage{courier}
\usepackage{dsfont} 
\usepackage{multicol}
\usepackage{tikz}
\usetikzlibrary{arrows,shapes,automata,backgrounds,petri}
\usepackage{listings} \lstset{numbers=left, numberstyle=\tiny, numbersep=5pt} \lstset{language=Perl} 

\pagestyle{fancy} %eigener Seitenstil
\fancyhf{} %alle Kopf- und Fußzeilenfelder bereinigen
%\fancyhead[L]{\doctitle  \nummer} %Kopfzeile links
\fancyhead[L]{\docauthors} %Kopfzeile rechts
\renewcommand{\headrulewidth}{0.5pt} %obere Trennlinie
\fancyfoot[L]{\doctitle  \nummer} %Kopfzeile links
\fancyfoot[R]{Seite \thepage} %Fußzeile rechts
\renewcommand{\footrulewidth}{0.5pt} %untere Trennlinie

% Titelseite Block oben
\newcommand{\titelblock}{
\sloppy
\begin{center}
\sffamily
{\Large{\veranstaltung \\}}
{\Huge{\doctitle  \nummer}\\}
\vspace{0.5cm}
\tutorium \\
\hrulefill
\end{center}
}

\newcommand{\docauthors}{Felix Wiedemann, Marcel Hellwig, Nasif Yueksel, David Weber, Sven-Hendrik Haase, Alias Dammer, Felix Ortmann}
\newcommand{\docdate}{\today}
\newcommand{\doctitle}{Ideensammlung}
\newcommand{\nummer}{ 1} %Übungszettelnummer hier anpassen
\newcommand{\tutorium}{WS 2013/2014}
\newcommand{\veranstaltung}{Projekt: Mikrorechner}
\newcommand{\gruppe}{}


\begin{document}
\titelblock

\section{Ideensammlung - Instruction Set Architecture}

\subsection{General}

\begin{multicols}{2}
\begin{itemize}

\item Wortbreite 32 bit
\item Adressbreite 32 bit
\item 3-Operanden (1 Ziel-, 2 Quell-Operanden)
\item 14 General-Purpose Register
\item \$0-Register: Null
\item \$15-Register: Return Address
\item 4 Statusflags (Z,N,C,V)
\item Big Endian
\item Befehle können Statusflags setzen
\item 16-Bit Immediates

\end{itemize}

\subsection{Arithmetik / Logik}

\begin{itemize}
\item ADD, ADC
\item SUB, SBC, RSB, RSC
\item MUL (MULL)
\item LSL, ADR, LSR, ROR 
\item AND, OR, XOR, NOT
\item SWAP (Wenn Felix ganz lieb fragt evtl soch)
\end{itemize}

\subsection{Controlflow}

\begin{itemize}
\item JMP
\item Jxx (Flag dependant JMP)
\item CALL
\item RET (== JMP \$15)
\end{itemize}

\subsection{Memory}

\begin{itemize}
\item LD (candy: load mit imm. add in one op)
\item STR
\item ToDo: indirekte Adresssierung
\end{itemize}
\end{multicols}

%\newline
%\hline

\subsection{Befehlsstruktur}
\[Opcode - (7bit) : R_{dest} - (4bit) : R_{s1} - (4bit) : OP2 - (17bit)\]
\[ALU - 0 : OpC0d3 - (5bit) : SF - (1bit)\]
\[MEM - 10 : LD/ST - (1bit) : R_{dest} - (4bit) : OP2 - (25bit, 1bit f. Reg / Imm.)\]
\[Jxx - 110 : Condition - (4bit) : OP2 - (25bit, 1bit f. Reg / Imm.)\]
\[JMP/CALL - 1110000 : OP2 - (25bit, 1bit f. Reg / Imm.)\]

\subsection{Fancy Magic}

\begin{itemize}
\item Divisionseinheit
\item Fließkomma
\item Touchscreen / Display (Mäder: das is ganz eklig) - flieht ihr Narren!!
\item Cache
\item SIMD
\end{itemize}


\subsection{Other Stuff, nice-to-haves, not-to-be-dones}

\begin{itemize}
\item read / write from somewhere
\item Software Interrupts
\item MMU (Kommentar aus dem Plenum: unnötig und scheiße)
\item OS-Aufrufe (falls wir mal eins haben)
\item GCC und lib-C portieren
\item Linux portieren
\item SD-Karte nutzen (wie Rasp-Pi?? als cache??)
\end{itemize}

\subsection{Codierung der Befehle - Binär}
Arithmetisch/Logische Befehle fangen mit 0 an
\begin{multicols}{2}
\emph{Präfix 00 + {Op-C0d3}}
\begin{itemize}
\item ADD 000
\item ADC 001
\item SUB 100
\item SBC 101
\item RSB 110
\item RSC 111
\end{itemize}

\emph{Präfix 100 + {Op-C0d3}}
\begin{itemize}
\item MUL 00
\item MULL 01
\item DIV 10
\end{itemize}

\emph{Präfix 010 + {Op-C0d3}}
\begin{itemize}
\item AND 00
\item OR 01
\item XOR 10
\item NOT 11
\end{itemize}


\emph{Präfix 011 + {Op-C0d3}}
\begin{itemize}
\item LSL 00
\item ASR 01
\item LSR 10
\item ROR 11
\end{itemize}


\end{multicols}



\end{document}
