\documentclass[a4paper]{article}

\usepackage[ngerman]{babel}
\usepackage[utf8]{inputenc} 
\usepackage{amsmath}
\usepackage{amssymb}
\usepackage{fancyhdr}
\usepackage{graphicx}
\usepackage{pdfpages}
\usepackage[left=2cm,right=2cm,top=2cm,bottom=3cm,includeheadfoot]{geometry}
\usepackage[scaled=.90]{helvet}
\usepackage{courier}
\usepackage{dsfont} 
\usepackage{tikz}
\usetikzlibrary{arrows,shapes,automata,backgrounds,petri}
\usepackage{listings} \lstset{numbers=left, numberstyle=\tiny, numbersep=5pt} \lstset{language=Perl} 

\pagestyle{fancy} %eigener Seitenstil
\fancyhf{} %alle Kopf- und Fußzeilenfelder bereinigen
%\fancyhead[L]{\doctitle  \nummer} %Kopfzeile links
\fancyhead[L]{\docauthors} %Kopfzeile rechts
\renewcommand{\headrulewidth}{0.5pt} %obere Trennlinie
\fancyfoot[L]{\doctitle  \nummer} %Kopfzeile links
\fancyfoot[R]{Seite \thepage} %Fußzeile rechts
\renewcommand{\footrulewidth}{0.5pt} %untere Trennlinie

% Titelseite Block oben
\newcommand{\titelblock}{
\sloppy
\begin{center}
\sffamily
{\Large{\veranstaltung \\}}
{\Huge{\doctitle  \nummer}\\}
\vspace{0.5cm}
\tutorium \\
\hrulefill
\end{center}
}

\newcommand{\docauthors}{Felix Wiedemann, Marcel Hellwig, Nasif Yueksel, David Weber, Sven-Hendrik Haase, Alias Dammer, Felix Ortmann}
\newcommand{\docdate}{\today}
\newcommand{\doctitle}{Ideensammlung}
\newcommand{\nummer}{ 1} %Übungszettelnummer hier anpassen
\newcommand{\tutorium}{WS 2013/2014}
\newcommand{\veranstaltung}{Projekt: Mikrorechner}
\newcommand{\gruppe}{}


\begin{document}
\titelblock

\section{Ideensammlung}

\subsection{Instruction Set Architecture}

\begin{itemize}

\item Wortbreite 32 bit
\item Adressbreite 32 bit
\item 3-Operanden (1 Ziel-, 2 Quell-Operanden)
\item 14 General-Purpose Register
\item \$0-Register: Null
\item \$15-Register: Return Address
\item 4 Statusflags (Z,N,C,V)
\item Big Endian
\item Befehle können Statusflags setzen
\item 16-Bit Immediates

\end{itemize}

\subsection{Arithmetik / Logik}

\begin{itemize}
\item ADD, ADC
\item SUB, SBC, RSB, RSC
\item MUL (MULL)
\item LSL, ADR, LSR, ROR 
\item AND, OR, XOR, NOT
\item SWAP
\end{itemize}

\subsection{Controlflow}

\begin{itemize}
\item JMP
\item Jxx (Flag dependant JMP)
\item verschiedene Ideen für Jxx - kodierbar mit 3 (hardcode), 6, 7, mehr-bit (Flag Auslesen)
\item CALL
\item RET (== JMP \$15)
\end{itemize}

\subsection{Fancy Magic}

\begin{itemize}
\item Divisionseinheit
\item Fließkomma
\item Touchscreen / Display
\item Cache
\item SIMD
\end{itemize}

\subsection{Memory}

\begin{itemize}
\item LD (candy: load mit imm. add in one op)
\item STR
\item ToDo: indirekte Adresssierung
\end{itemize}

\subsection{Other Stuff, nice-to-haves, not-to-be-dones}

\begin{itemize}
\item read / write from somewhere
\item Software Interrupts
\item MMU (Kommentar aus dem Plenum: unnötig und scheiße)
\item OS-Aufrufe (falls wir mal eins haben)
\item Windows portieren (hier werden leute sterben)
\item alles vorher mit 0 / 1 modellieren
\item SD-Karte nutzen (wie Rasp-Pi?? als cache??)
\end{document}
