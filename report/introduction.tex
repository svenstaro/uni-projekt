% Einleitung 
Ein Computer ist keine komplizierte Maschine. Die Meisten wissen, dass er aus einem Prozessor, einem Arbeitsspeicher und einer Festplatte besteht. Der Prozessor steuert die Maschine und treibt die Datenverarbeitung voran, der Arbeitsspeicher dient als Puffer für die Aufträge und die Festplatte ist für die persistente Speicherung der Daten zuständig. Weiterhin können noch Hardwareteile wie eine Grafikkarte, ein DVD-Laufwerk oder eine Webcam verbaut sein.
Beschäftigt man sich etwas genauer mit dem Thema, wird man schnell feststellen, dass nicht all diese Komponenten zwingend notwendig sind. Dass die Webcam oder das DVD-Laufwerk nicht unbedingt zu jedem PC gehören erkennen wohl die Meisten. Dass auch die Grafikkarte und die Festplatte nicht zur Grundidee eines Computers gehören ist nicht so offensichtlich.

Der österreichisch-ungarische Mathematiker John von Neumann publizierte im Jahr 1945 eine Architektur, die die Basis moderner PCs, Laptops, Smartphones darstellt.
Die nach ihm benannte Architektur besteht aus einer Recheneinheit, der Central Processing Unit (CPU), einem Speicher für die Daten und Befehle, einer Ein- und Ausgabe-Einheit, damit der Mensch mit der Maschine interagieren kann und einem Bussystem, das die genannten Komponenten miteinander verbindet. Die CPU steuert dabei alle anderen Komponenten. Aber woher weiß die CPU, was zu tun ist? Denn der Prozessor ist auch nur ein Hardwareteil, das ohne Anweisungen nicht weiß, was zu tun ist. Ein Algorithmus muss die Regeln vorgeben, wie Befehle auszuführen sind. Dieser Algorithmus bestimmt, wie schnell und effizient Befehle ausgeführt werden. Der Prozessor kann noch so modern sein, wenn der Algorithmus schlecht ist, wird auch der Computer keine Wunder vollbringen. Nicht umsonst verdienen Unternehmen wie Intel, AMD und neuerdings auch Qualcomm Milliarden mit dem Verkauf von Prozessoren. 

Wie wichtig der Algorithmus ist, sieht man auch gut am Beispiel von Apple. Das Top-Smartphone von Apple läuft mit einem 1,3 GHz-Dual-Core-Prozessor. Das Top-Modell von Samsung läuft mit einem 2,5 GHz-Quad-Core-Prozessor. Dennoch laufen beide Smartphones mit einem ähnlichen Arbeitstempo und erreichen ähnliche Werte in Benchmarks. Natürlich kann man die 2 Prozessoren nicht direkt miteinander vergleichen, da es sich um andere Architekturen und Hersteller handelt. Außerdem läuft nicht das gleiche System auf beiden Geräten. Dennoch ist der Unterschied der Taktfrequenz deutlich.

Mit diesem \glqq magischen\grqq \ Algorithmus durften und mussten wir uns in dem Projekt \glqq Entwurf, Realisierung und Programmierung eines Mikrorechners\grqq \ beschäftigen. 

\section{Das Team}
Unser Team besteht aus verschiedensten technikbegeisterten Mitgliedern. Zum einen aus Nasif Yüksel, Student der Wirtschaftsinformatik im fünften Semester. Er ist der strahlende Stern des Testens in unserem Projekt. Marcel Hellwig, Student der Informatik im fünften Semester. Der unbestrittene König der Hardware. David Weber, Student der Wirtschaftsinformatik. Willenloser Sklave in der Softwarefraktion. Felix Wiedemann, Student der Informatik im fünften Semester. Er ist der Anführer der Softwarerebellen. Felix Ortmann, Student der Informatik im sechsten Semester. Er war der Verfechter der Konventionen.\todo{Formulierung sehr fluffig aber schon iwie witzig ;)}

\section{Motivation}
Zum einen waren wir einfach neugierig zu erfahren, wie die Software mit der Hardware im Detail zusammenspielt. In der Veranstaltung Rechnerstrukturen haben wir ersten Kontakt mit diesem Thema bekommen. Um unsere großes Interesse weiter zu vertiefen und in die Praxis umzusetzen, hat sich dieses Projekt ideal angeboten.
Ein weiterer Grund für dieses Projekt war, dass wir unbedingt unser Projekt im Arbeitsbereich TAMS machen wollten. Denn bereits im Seminar wurden wir durch interessante Themen und einer tollen Unterstützung durch das Lehrpersonal überrascht.
Außerdem hat die Vorstellung einen eigenen Mikrorechner zu konzipieren, jeden von uns gereizt.

\section{Material}
Um unsere Testergebnisse sichtbar zu machen, wurde ein FPGA-Prototyp, u.a. ausgestattet mit 4 LEDs, 4 Tastern, einem 800x400-Touchscreen und diversen Schnittstellen. Unser Ziel war es die LEDs durch drücken der Taster zum leuchten zu bringen. % Um unseren Code auf das Gerät zu bringen haben wir die serielle Schnittstelle benutzt. FALSCH

Ob wir unser Ziel erreicht haben, was besonders gut oder schlecht lief, auf was für Schwierigkeiten wir gestoßen sind und welche Herausforderungen wir gemeistert haben,wird auf den nächsten Seiten detailliert darstellen. %nicht von wir sprechen imt text. 
